\documentclass[titlepage, 11pt, a4paper]{article}
\usepackage[brazil]{babel}
\usepackage[latin1]{inputenc}
\usepackage[dvips]{graphicx}
\usepackage{verbatim}
\usepackage{subfigure,epsfig}
\usepackage{amsmath,amsthm,amssymb,amscd}
\usepackage{fancyhdr}
\usepackage{amscd}
\usepackage{booktabs}
\usepackage{float}
\title{ \bf }

\usepackage{color, float, bbm, multicol, xypic}
								                                                                                
\hoffset = 0pt % 1inch + hoffset (margem vertical esquerda)
\voffset = 0pt % 1inch + voffset (margem horizontal superior)
\oddsidemargin = 0pt % complemento da margem vertical esquerda
\topmargin = 0pt % complemento da margem horizontal superior
\topmargin = 0pt % complemento da margem horizontal superior
\headheight = 12pt % altura do header
\headsep = 25pt % distancia do header ate o texto
\textheight = 650pt % altura do texto
\textwidth = 451pt % largura do texto
\marginparsep = 0pt % distancia do texto ate a
\marginparwidth = 0pt % largura da margin notes (54pt)
\paperheight = 297mm % a4
\paperwidth = 210mm % a4
                                                                              
%---TITULO---
\title{\huge{Transposi�ao de distribui�ao Quadrada em distribui�ao normal}}
\author{{\bf Fernando Vicente Marques Favini} \\
\footnotesize{Estudos para R-Type}}
\date{\today}
																								                                                                                
%--- DEFINICOES ---
\numberwithin{equation}{section}
\numberwithin{table}{section}
\numberwithin{figure}{section}
					                                                                                
%--- O DOCUMENTO EM SI ---
\begin{document}
                                                               
%--- ESTILO DA PAGINA ---
\pagestyle{fancy} % fancy,plain,empty,headings
\lhead{Transposi�ao de distribui�ao Quadrada em distribui�ao normal}
\chead{}
\rhead{\leftmark} % numero e titulo da section
\lfoot{Estudos para R-Type}
\cfoot{}
\rfoot{\thepage} % numero da pagina
\renewcommand{\headrulewidth}{0.4pt}
\renewcommand{\footrulewidth}{0.4pt}
                                                                               
%\pagenumbering{roman} % no comeco, paginas em romano
					                                                                                
%--- SUMARIOS ---
\maketitle
\thispagestyle{plain} % sem barras no header na capa (PAGINA 1)
\hrule % linha estetica
\tableofcontents
\listoftables
\listoffigures
%\hrule % nova linha estetica
\newpage                                                                               
%--- RESUMO ---
\pagenumbering{arabic} % agora, numeracao padrao
\section{Introdu\c{c}\~{a}o}

Podemos utilizar a estat\'{i}stica para entender eventos binomiais, que s\~{a}o os modelos
em que duas alternativas s\~{a}o apresentadas, sucesso e falha.

Alguns modelos binomiais comuns no universo de simula\c{c}\~{a}o
s\~{a}o aqueles em que um modelo simula um evento com uma probabilidade de acerto.
Um numero aleat\'{o}rio \'{e} gerado
e comparado com um lim\'{i}te, e caso este valor passe ou
fique abaixo deste lim\'{i}te o evento \'{e} considerado um sucesso ou uma falha.

Em um universo discreto, este numero seria o equivalente a um lan\c{c}amento de um dado,
onde um numero \'{e} utilizado como par\^{a}metro de sucesso, como por exemplo, o número limite 16
em um lan\c{c}amento de um dado de 20 faces garante uma taxa de sucesso de $25\%$

Para uma simula\c{c}\~{a}o, utilizamos geradores aleat\'{o}rios de dsitribui\c{c}\~{a}o quadrada, onde
todos os valores tem a mesma probabílidade de serem lan\c{c}ados
de uma distribui\c{c}\~{a}o quadrada (figura \ref{intro:squared}), variando entre $0$ e $1$
e comparamos este valor contra um limite de sucesso do dito evento.

Repetimos o procedimento \introexperimentsize{} vezes para cada experimento obtendo um indice de sucesso,
e repetindo o experimento \introexperimentrepeats{} vezes gerando um histograma de indices obtendo o gr\'{a}fico
\ref{intro:binomial}

\begin{center}
\begin{figure}[H]
\begin{center}
\label{intro:squared}
\epsfig{file=../eps/transpose/squared.eps, scale=1, angle=0}
\caption{Distribui\c{c}\~ao quadrada}
\end{center}
\end{figure}
\end{center}

\begin{center}
\begin{figure}[H]
\begin{center}
\label{intro:binomial}
\epsfig{file=../eps/transpose/binomial.eps, scale=1, angle=0}
\caption{Distribui\c{c}\~ao de experimento bin\^omial}
\end{center}
\end{figure}
\end{center}

\end{document}

