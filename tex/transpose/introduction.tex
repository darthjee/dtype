\chapter{Introdu\c{c}\~{a}o}
Podemos utilizar a estat\'{i}stica para entender eventos binomiais, que s\~{a}o os modelos
em que duas alternativas s\~{a}o apresentadas, sucesso e falha.

Alguns modelos binomiais comuns no universo de simula\c{c}\~{a}o
s\~{a}o aqueles em que um modelo simula um evento com uma probabilidade de acerto,
e um numero aleat\'{o}rio \'{e} gerado e comparado com um lim\'{i}te, e caso este valor passe ou
fique abaixo deste lim\'{i}te o evento \'{e} considerad um sucesso ou uma falha.

Em um universo discreto, este numero seria o equivalente a um lan\c{c}amento de um dado,
onde um numero \'{e} utilizado como par\^{a}metro de sucesso, como por exemplo, o número limite 16
em um lan\c{c}amento de um dado de 20 faces garante uma taxa de sucesso de $25\%$

Para uma simula\c{c}\~{a}o, utilizamos geradores aleat\'{o}rios de dsitribui\c{c}\~{a}o quadrada, onde
todos os valores tem a mesma probabílidade de serem lan\c{c}ados, variando entre $0$ e $1$
e comparamos este valor contra a taxa de sucesso do dito evento.

Repetimos o procedimento 100 vezes para cada experimento obtendo um indice de sucesso,
e repetindo o experimento 100 vezes gerando um histograma de indices obtendo o gr\'{a}fico
