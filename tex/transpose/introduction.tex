\section{Introduction}

We may use statistics to understand binomial events that are models where
two alternatives are presented, option A or option B.

Some common binomial models are, being born a female or male, having or not an specif genetic trait,
or even a success or failure on the toss of a coin or dice, which can usually
be simulated with the generation of a random number (equivalent of the toss of a dice in the discrete universe)
and then check it against a threashold that would define it as being
option A or B, or success or failure if you will.

An example would be the toss of a $20$ sided dice and set the threashold as $10$ so any value higher
than that is considered a success giving us a $50\%$ success rate.

In order to simulate this toss, we use a square probability distribuited (fig \ref{intro:squared}) number generator
where every number has the same probability of comming out, varaying from the range of 0 to 1
then compare it with the equivalent threashold ($0.5$ for a $50\%$ success rate) to determin if an event was
a success or failure.

We repete the procediment \introexperimentsize{} times so that we can obtein the given success rate of that experiment,
which may be different from the theoretical expected success rate, but by repeating the experiment
\introexperimentrepeats{} times we can build an histogram of success rate which should have it's peak around the
expected success rate.
\ref{intro:binomial}

\begin{center}
\begin{figure}[H]
\begin{center}
\epsfig{file=../eps/transpose/squared.eps, scale=1, angle=0}
\caption{Squared distribuition}
\label{intro:squared}
\end{center}
\end{figure}
\end{center}

\begin{center}
\begin{figure}[H]
\begin{center}
\epsfig{file=../eps/transpose/binomial.eps, scale=1, angle=0}
\caption{Binomial experiment distribuition}
\label{intro:binomial}
\end{center}
\end{figure}
\end{center}

This distribuition of probability seen in \ref{intro:binomial} is known as gaussian or normal distribuition
where the probability has it's peak at the average

\begin{center}
\begin{equation}
  g(x)
\label{intro:gaussian}
\end{equation}
\end{center}
